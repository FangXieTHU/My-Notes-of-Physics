\documentclass{article}

\usepackage[left=1.5cm, right=1.5cm, top=3cm, bottom = 3cm]{geometry}

\usepackage{amsmath}
\usepackage{amsfonts}
\usepackage{amssymb}
\usepackage{graphicx}
\usepackage{float}
\usepackage{indentfirst}
\usepackage{wrapfig}
\usepackage{latexsym}
\usepackage{hyperref}
\usepackage{feynmf}
\linespread{1.1}

% frequently used 3-vector notations
\newcommand{\mtp}{\mathbf{p}}
\newcommand{\mtq}{\mathbf{q}}
\newcommand{\mtk}{\mathbf{k}}
\newcommand{\pnx}{\mathbf{x}}
\newcommand{\pny}{\mathbf{y}}
\newcommand{\pnr}{\mathbf{r}}

% frequently use spin notations

\newcommand{\uspin}{\uparrow}
\newcommand{\dspin}{\downarrow}

\author{Fang Xie, \emph{Department of Physics, Tsinghua University}}
\title{{\bf{An Introduction to Second Quantization and Path Integral in Statistical Mechanics}}}
\date{\today}

\begin{document}
\maketitle
\section{Introduction}
Second Quantization is the best way to describe the many-body quantum systems. We can get a lot of interesting results from second quantization, such as the band structure of graphene and Ferromagnetic coupling. We can also construct the path integral method of Grand Canonical Partition function. The path Integral and Green function can help us to get the thermal quantities of 

\section{Second Quantization}
\subsection{Defination}
\subsection{One-body operator}
\subsection{Two-body operator}

\section{Some Examples}

\section{Coherent States for Bosons}

\section{Grassmann Number and Coherent States for Fermions}
\subsection{Grassmann Number}
\subsection{Coherent States}

\section{Path Integral Representation of Grand Canonical Partition Function}

\section{Matsubara Representation, Green Function and Bose/Fermi Distribution}
\subsection{Matsubara frequencies}
\subsection{Ensemble Average and Green Function}
\subsection{Bose/Fermi Distribution}

\end{document}











