\documentclass{article}

\usepackage[left=1.5cm, right=1.5cm, top=3cm, bottom = 3cm]{geometry}

\usepackage{amsmath}
\usepackage{amsfonts}
\usepackage{amssymb}
\usepackage{graphicx}
\usepackage{float}
\usepackage{indentfirst}
\usepackage{wrapfig}
\usepackage{latexsym}
\usepackage{hyperref}
\usepackage{feynmf}
\usepackage{eufrak}
\linespread{1.1}

% frequently used 3-vector notations
\newcommand{\mtp}{\mathbf{p}}
\newcommand{\mtq}{\mathbf{q}}
\newcommand{\mtk}{\mathbf{k}}
\newcommand{\pnx}{\mathbf{x}}
\newcommand{\pny}{\mathbf{y}}
\newcommand{\pnr}{\mathbf{r}}
\newcommand{\Tr}{\mathrm{Tr}}

% frequently use spin notations

\newcommand{\uspin}{\uparrow}
\newcommand{\dspin}{\downarrow}

\author{\emph{Fang Xie, Department of Physics \& IASTU, Tsinghua University}}
\title{{\bf{Reading Note of Majorana Fermion Positivity}}}
\date{\today}
\begin{document}
\maketitle
\section{Introduction}
In this note I will talk about the proof of {\bf{reflection positivity of Majoranas}}, and the conditions of the theorem. Finally I will talk about some examples, including Ising model and Heisenberg Model.

The theorem we want to prove is:
$$
\Tr{\left[A\vartheta(A)e^{-H}\right]}\geq 0
$$
if the Hamiltonian $H$ satisfy some conditions. Operator $\vartheta$ is anti-linear. In the proof we can find out the condition. We will talk about these in detail in the following sections.
\section{Proof}
The proof of the theorem can be divided into the following parts:
\begin{enumerate}
\item[(1)] Define the (Hilbert) space and algebra we are consider about.
\item[(2)] Study the monomial basis on our algebra.
\item[(3)] Study the Hamiltonian of the problem.
\item[(4)] Prove the reflection positivity of an inner-product.
\item[(5)] Use the result of (4) to get the main result.
\end{enumerate}
Now the previous steps will be shown in detail in the following sections.
\subsection{Definition of the space and algebra}
First we have to deal with the concept ``Majorana''. As an example, we can consider about a lattice, and the creation-annihilation operators can be decomposed into to parts, say two Majoranas:
$$
c_{2j-1} = a_j + a^\dagger_j,\quad c_{2j} = i(a_j-a^\dagger_j)
$$
Thus it is easy to find the Majoranas obey the Clifford Algebra:
\begin{equation}
\{c_i,c_j\} = 2\delta_{ij}
\end{equation}
Now we consider the geometrical significants of these operators as a lattice $\Lambda$, and define a reflection operator $\vartheta$, by which the lattice is divided into two sectors:
$$
\Lambda = \Lambda_+\cup\Lambda_-\,,\Lambda_+\cap \Lambda_- = \O\,,\vartheta(\Lambda_+) = \vartheta(\Lambda_-)
$$
Now we need the operator $\vartheta$ is anti-linear which means that
$$
\theta(f+ \lambda g) = \vartheta(f) + \bar{\lambda}\vartheta(g)\,,\lambda \in \mathbb{C}
$$
in which $f$ and $g$ are vectors in the Hilbert space. In other words, the operator will take the complex conjugate of the target vector.

Now we will introduce some of the space and algebra. $\mathfrak{A}(\mathcal{B})$ denotes the algebra defined on the set $\mathcal{B}\subset \Lambda$; $\mathfrak{A}_\pm = \mathfrak{A}(\Lambda_\pm)$ is the algebra on $\Lambda_\pm$; $\mathfrak{A}^\mathrm{even}_\pm$ is the algebra with even power on $\Lambda$. In conclusion, the definition is
\begin{eqnarray*}
&&\forall A \in \mathfrak{A}_-, A = \sum a_\beta M_\beta, M_\beta \textrm{ is the product of an arbitrary number of Majoranas }c_j \in \Lambda_- \\
&&\forall A \in \mathfrak{A}^{\mathrm{even}}_-, A = \sum a_\beta M_\beta, M_\beta \textrm{ is the product of an even number of Majoranas }c_j \in \Lambda_- 
\end{eqnarray*}
That is all we need to define before we talk about the problem.

\subsection{Hamiltonian}
We hope the Hamiltonian can be written as the following form:
\begin{equation}
H = H_- + H_0 + H_+
\end{equation}
and we hope $H_- = \vartheta(H_+)$. The $H_0$ term will have the following form:
\begin{equation}
H_0 = \sum_{\mathfrak{J}}J_{\mathfrak{J}}i^{\sigma(\mathfrak{J})}C_{\mathfrak{J}}\vartheta(C_{\mathfrak{J}})
\end{equation}
each $\mathfrak{J}$ denotes a distinct product of many Majoranas $C_{\mathfrak{J}}$, and $\sigma(\mathfrak{J})$ is defined as:
$$
\sigma(  \mathfrak{J} ) = n(\mathfrak{J})\,\mathrm{mod}\,2
$$
in which $n(\mathfrak{J})$ is the number of Majoranas in the product $C_\mathfrak{J}$. At the end of this section we will see the coefficients $J_\mathfrak{J}$ satisfy the following condition if we want reflection positivity is correct:
\begin{eqnarray*}
\left\{\begin{array}{ll}
\textrm{all} J_\mathfrak{J} \geq 0 & \sigma(\mathfrak{J}) = 1\\
\textrm{or all} J_\mathfrak{J} \leq 0 &\\
\textrm{all} J_\mathfrak{J} \leq 0 & \sigma(\mathfrak{J}) = 0
\end{array}\right.
\end{eqnarray*}
The reason why these inequalities should be satisfied will be shown in section 2.4.

\subsection{Reflection positivity of the inner-product}
Consider about an operator $A\in \mathfrak{A}(\Lambda)$, it can be written as a linear combination of monomial basis:
\begin{equation}
A = \sum_\beta a_\beta M_\beta
\end{equation}
in which $M_\beta$ is a monomial. So easily we can find there are $2^N$ different monomials in $\mathfrak{A}(\Lambda)$.

\subsection{The reflection positivity of Majoranas}

\subsection{Reflection Bounds (Generalized Cauchy-Schwartz inequility)}

\section{Ising Model and Heisenberg Model as examples}


\begin{thebibliography}{99}
\bibitem{a}{Jaffe, A., \& Pedrocchi, F. L. (2014). Reflection Positivity for Majoranas. Annales Henri Poincaré, 16(1), 189–203. http://doi.org/10.1007/s00023-014-0311-y}

\end{thebibliography}
\end{document}








