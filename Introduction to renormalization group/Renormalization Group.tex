\documentclass{article}

\usepackage[left=1.5cm, right=1.5cm, top=3cm, bottom = 3cm]{geometry}

\usepackage{amsmath}
\usepackage{amsfonts}
\usepackage{amssymb}
\usepackage{graphicx}
\usepackage{float}
\usepackage{indentfirst}
\usepackage{wrapfig}
\usepackage{latexsym}
\usepackage{hyperref}
\usepackage{feynmf}
\linespread{1.1}

% frequently used 3-vector notations
\newcommand{\mtp}{\mathbf{p}}
\newcommand{\mtq}{\mathbf{q}}
\newcommand{\mtk}{\mathbf{k}}
\newcommand{\pnx}{\mathbf{x}}
\newcommand{\pny}{\mathbf{y}}
\newcommand{\pnr}{\mathbf{r}}

% frequently use spin notations

\newcommand{\uspin}{\uparrow}
\newcommand{\dspin}{\downarrow}

\author{\emph{Fang Xie, Department of Physics \& IASTU, Tsinghua University}}
\title{{\bf{An Introduction to Renormalization Group and Critical Phenomeon}}}
\date{\today}
\begin{document}
\maketitle
\section{Introduction}
In modern physics, the concept of ``low-energy effective theory'' is very important.

\section{General Theory of Renormalization Group}
\subsection{Basic Steps}

\subsection{$\beta$-function}

\subsection{Fixed point and renormalization flow}

\subsection{Scaling function and critical exponents}

\section{Example 1. Ising Model and $\phi^4$ Theory: One-loop renormalization}
\subsection{Partition Function}

\subsection{One-loop renormalization}

\subsection{Rescaling and the critical exponents of Ising Model}

\section{Example 2. Kosterlitz-Thouless Transition and Topological Excitations}
\subsection{Classical XY model}

\subsection{The vortex excitation}

\subsection{First order renormalization}

\section{Example 3. Dilute Bose Gas and its renormalization analysis}

\end{document}















