\documentclass{article}
\title{\bf Note: Quantum Physics in One Dimension}
\author{Fang Xie}
\date{\today}

%\usepackage[left=1.5cm, right=1.5cm, top=3cm, bottom = 3cm]{geometry}

\usepackage{amsmath}
\usepackage{amsfonts}
\usepackage{amssymb}
\usepackage{graphicx}
\usepackage{float}
\usepackage{wrapfig}
\usepackage{latexsym}
\usepackage{hyperref}
\usepackage{feynmf}
\usepackage{exscale}
\usepackage{relsize}
\linespread{1.1}
\begin{document}
\maketitle
\section{Basic Concepts fot interacting quantum systems}
\subsection{Weak Interacting particles}
\subsubsection{Fermi liquid and free fermions}
Anti-symmetric wave function: $c^\dagger_k(c_k)$ the (de)creation operator be anti commute with each other. Hamiltonian
$$
H = \sum_k \epsilon_k c^\dagger_kc_k
$$
and the Fermi energy in condensed matter system is about $\epsilon_F \sim 1\mathrm{eV}\sim 12000\mathrm{K}$. For cold atom systems, $\epsilon_F \sim 100\mathrm{nK}$. 
Density of states:
$$
D(E) = \sum_k \delta(E-\epsilon_k)
$$
Average:
$$
\langle \cdots \rangle = Z^{-1}\mathrm{Tr}\left[e^{-\beta H}\cdots\right]
$$
Spechual Function. The correlation function:
$$
\langle \mathrm{GS} | e^{iHt}c_{x,t}e^{-iHt}c^\dagger_{0,0}|\mathrm{GS}\rangle
$$
Green Function:
$$
G(x,t)=-i\theta(t)\langle [c_{x,t},c^\dagger_{0,0}]\rangle
$$
Probability to find an excitation:
$$
A(k,\omega) = -\frac{1}{\pi}\mathrm{Im}{G(k,\omega)}
$$
(Kallen-Lehmmann)
and the Green Function is
$$
G(k,\omega) = \frac{1}{\omega-\xi_k+i\delta}\quad \delta = 0^+
$$
$\uparrow$ is the retarded Green Function.\\
Effect of interaction in metals $U\sim E_F$ so the fermi surface no longer exist.
Landau Quasi particles--dressed electrons, redefine parameters, effective mass, free quasi particles. $n(\epsilon)$ of quasi particles has a jump at $E_F$.

In 2d and 3d, FLT can get rid of interaction.

\subsubsection{Interacting Boson system}
BEC:$\langle b^\dagger_k b_k\rangle = N \delta_{k,0}$ and $n_{k} = N_0\delta_{k,0}+ n^{\mathrm{regular}}(k)$.\\Bogoliubov approximation:
$$
H = \sum_k \xi_k b^\dagger_k b_k + U\sum_{kk'q}b^\dagger_{k+q}b^\dagger_{k'-q}b_{k'}b_{k}
$$
assume that there is macroscopic occupation in $k = 0$ and the Hamiltonian will be quadratic.

\subsection{Strong Correlation System}
A theory on lattice, tight binding model.
$$
H = -t\sum_{\langle ij\rangle}c^\dagger_i c_j
$$
transform into momentum space and the result will be
$$
H = \sum_k (-2t\cos{k})c^\dagger_kc_k
$$
in which the lattice constant $a = 1$. If we add an onsite interactiong term $U$ (Mott-Hubbard Model), it will be come a insulator.
\subsection{Hubbard Model}
This model is crucial to understand the strong correlated Bose system.
The Hamiltonian for Bosons has hooping terms between nearest neighbours and onsite interaction:
$$
H = -t\sum_{\langle ij\rangle}(c^\dagger_i c_j + \mathrm{h.c.})+ U\sum_i \rho_i(\rho_i-1) - \mu\sum_i\rho_i
$$
for Fermions we have:
$$
H = -t\sum_{\langle ij\rangle}(c^\dagger_i c_j + \mathrm{h.c.})+ U\sum_i \rho_{i\uparrow}\rho_{i\downarrow} - \mu\sum_i\rho_i
$$
The two Hamiltonians is called Hubbard Model(1963).\\
{\bf{Bose-Hubbard Model}}. In the case Bosons without spin. The Hamiltonian on the lattice is
$$
H = -t\sum_{\langle ij\rangle} b^\dagger_i b_j + \frac{U}{2}\sum_{i}\rho_i(\rho_i-1)
$$
Mott-Insulatir phase transition is neither first nor second order phase transition. For $U\gg t$, the system is in the Mott insulator phase, and for $U\ll t$, the system is in the metal phase(in high dimensions, it's Fermi liquid; in one dimension, it's Tomonaga-Luttinger liquid). Mott insulator has one electron in each site, while Band insulator has two electrons ion each site
\\{\bf{Superexchange}}. For Fermions with Hubbard on site interaction, an effective Hamiltonian of spin degree of freedom will be:
$$
H^\mathrm{spin} = J\sum_{\langle ij\rangle}\vec{S}_i\cdot\vec{S}_j
$$
and the parameter $J$ is of the order $J\sim \frac{4t^2}{U}$. {\bf{Superexchange for spin one-half Fermions is the antiferromagnetic interaction.}} Because the virtual process with parallel spin on neighbour sites is banned by Pauli principle.\\
What about the bosons with spin degree of freedom? {\bf{Superexchange interaction with spin one-half Bosons will be ferromagnetic.}}\\
{\bf{Spinless Fermions}}. Since we cannot have the onsite interaction, we can add interaction on neighbour sites:
$$
H = -t\sum_{\langle ij\rangle}(c^\dagger_ic_j + \mathrm{h.c.}) + U\sum_{\langle ij\rangle}n_i n_j
$$
\section{One dimensional systems; Tomonaga-Luttinger liquid}
\subsection{Introduction}
Physics systems with reduced dimensionality: physics at edge, interface between $\mathrm{LaO}$ and $\mathrm{StO}$ is superconductor, but the two themselves are insulators. In one dimension, only collective excitations can exist. In one dimension the quantum fluctuation is too strong that the long range order will be zero (Mermin-Wagnar theorem). 
Some examples of one dimension systems: Organic conductors, carbon nanotubes, quantum wires, spin chains and ladders(PRL 86 5168, PRL 101 137207, PRB 79, 020408)\\
In Cold Atoms system, the kinetic and interaction is easy to manipulate. 

The Typical problem in continuum Quantum systems is
$$
H = \int dx \, \psi^\dagger\frac{-\nabla^2}{2m}\psi + \frac{1}{2}\int dxdy\,V(x-y)\rho(x)\rho(y) -\mu\int dx \, \rho(x)
$$
or in the lattice systems
$$
H = -J\sum_{\langle ij\rangle}b^\dagger_ib_j + U\sum_{i}n_i(n_i-1)-\mu\sum_in_i
$$
So how to treat these problems? \\Standard many-body theory.
\begin{itemize}
\item Exact solution (Bethe ansatz)
\item Field theory (bosonization, CFT)
\item Numeries (DMRG,MC, etc.)
\end{itemize}
\subsection{Luttinger Liquid}
In one dimension systems the perturbation theory failed so we need to use new methods to treat. The Hamiltonian is
$$
H = \int dx \, \psi^\dagger\frac{-\nabla^2}{2m}\psi + \frac{1}{2}\int dx\,\frac{g}{2} \rho(x)\rho(x) -\mu\int dx \, \rho(x)
$$
in which the interacting term is a $\delta$ potential. Now we can define a field $\phi_l$ to describe the particle distribution:
$$
\rho(x) = \sum_ i \delta(x-x_i) = \sum_n|\nabla\phi_l(x)|\delta(\phi_l(x)-2\pi n)
$$
and that is a unique way of labelling particles.
Define
$$
\phi_l(x) = 2\pi\rho_0 x -2\phi(x)
$$
and 
$$
\rho(x) = \left[\rho_0-\frac{1}{\pi}\nabla\phi(x)\right]\sum_{p\textrm{\,integer}} e^{i2p(\pi\rho_0x-\phi(x))}\quad \textrm{Poison summation formula}
$$
in which the field $\phi(x)$ varies slowly. Then how about the single particle operator?
$$
\psi^\dagger = [\rho(x)]^{1/2}e^{-i\theta(x)}
$$
we can find the commutation relation of these operators are
$$
\left[\frac{1}{\pi}\nabla\phi(x),\theta(x')\right] = -i\delta(x-x')
$$
in which $\theta(x)$ is the superfluid phase. The Hamiltonian then becomes
$$
H = \frac{1}{2\pi}\int dx\, \left[uK(\pi\Pi(x))^2+\frac{u}{K}(\nabla\phi(x))^2\right]
$$
in which the parametres are
$$
uK = \frac{\rho_0}{2m}
$$
and the explicit value of them are determined both by $U, m, \rho_0$ and some macroscopic values such as compressibility. Calculation result show that the correlation of Boson operators and density operators decays as power law.

For Fermions, the single particle operator is defined as(Jordan–Wigner transformation)
$$
\psi_F^\dagger = \psi_B^\dagger e^{i\frac{1}{2}\phi_l(x)}
$$
so the Fermion operator can be written as
$$
\psi^\dagger_F = \left[\rho_0-\frac{1}{\pi}\nabla\phi(x)\right]^{1/2}\sum_p e^{i(2p+1)(\pi\rho_0x-\phi(x))}e^{-i\theta(x)}
$$

\section{Experimental realizations}

\section{Disorder and other perturbation}

\section{Beyond Tomonaga-Luttinger liquid}

\end{document}






