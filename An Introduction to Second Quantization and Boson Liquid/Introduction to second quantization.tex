\documentclass{article}

\usepackage[left=1.5cm, right=1.5cm, top=3cm, bottom = 3cm]{geometry}

\usepackage{amsmath}
\usepackage{amsfonts}
\usepackage{amssymb}
\usepackage{graphicx}
\usepackage{float}
\usepackage{indentfirst}
\usepackage{wrapfig}
\usepackage{latexsym}
\usepackage{hyperref}
\usepackage{feynmf}
\linespread{1.1}

% frequently used 3-vector notations
\newcommand{\mtp}{\mathbf{p}}
\newcommand{\mtq}{\mathbf{q}}
\newcommand{\mtk}{\mathbf{k}}
\newcommand{\pnx}{\mathbf{x}}
\newcommand{\pny}{\mathbf{y}}
\newcommand{\pnr}{\mathbf{r}}

% frequently use spin notations

\newcommand{\uspin}{\uparrow}
\newcommand{\dspin}{\downarrow}

\author{Fang Xie, \emph{Department of Physics, Tsinghua University}}
\title{{\bf{An Introduction to Second Quantization and Path Integral in Statistical Mechanics}}}
\date{\today}

\begin{document}
\maketitle
\section{Introduction}
Second Quantization is the best way to describe the many-body quantum systems. We can use the creation-annihilation operators to get a lot of interesting results by this method. In this article I will follow the formalism of Pathria's book and show how to use second quantization to get the statistical properties of boson liquid and fermi liquid.

\section{Second quantization of Bosons and Fermions}
Actually second quantization is the quantization of fields. So we need to define the creation-annihilation operators of the boson or fermion field:
\begin{eqnarray}
[a_i,a^\dagger_j] &=& \delta_{ij}\\
\left[a_i,a_j\right] &=& 0 \\
\left[a^\dagger_i,a^\dagger_j\right] &=& 0 
\end{eqnarray}
in which $i,j$ means some eigenvalue of some operators. For Fermions the commutation relation is altered by anti-commutation relation. Since the commutation relation is just like the upper and lower operators of Harmonic oscillator, we can find that
$$
N = \sum_i a^\dagger_i a_i 
$$
is the operator of particle number.
Now introduce the {\bf{one-body operator}}: for any one-body operator $\mathcal{O}_1$, its second-quantized form under its eigenbasis is:
\begin{equation}
\hat{O} = \sum_{i}o_i a^\dagger_i a_i
\end{equation}
and $o_i$ is the eigenvalue of first-quantized operator $O_1$. Now if we do an representation transformation to basis that are not the eigenvectors of $O_1$, then the second quantized operator will be:
$$
\hat{O} = \sum_{ij}\langle i | O_1|j\rangle\, a^\dagger_i a_j
$$
For example, the second quantized Hamiltonian is
$$
H = \sum_{mtk\sigma}a^\dagger_{\mtk\sigma}a_{\mtk\sigma} \quad\textrm{Free particle system}
$$
or
$$
H = -t\sum_{\langle ij \rangle}a^\dagger_i a_j\quad \textrm{(Tight-binding approximation)}
$$

\section{}

\end{document}











